% !TEX root = ../main.tex

\chapter{Conclusiones}

El proyecto \textit{Rectificador de Imágenes} (QuadFix) ha logrado su objetivo principal: crear una herramienta intuitiva y eficiente para rectificar imágenes que contienen rectángulos mediante transformaciones proyectivas. A lo largo del desarrollo, se implementaron algoritmos de visión por computadora y se diseñó una interfaz gráfica amigable que facilita la interacción con el usuario. Además, se incorporaron funcionalidades importantes como el ajuste dinámico de proporciones y el soporte para modos visuales claros y oscuros, lo que contribuye a mejorar tanto la adaptabilidad como la experiencia general del usuario.

\section{Logros Principales}

En cuanto a los resultados obtenidos, el sistema es capaz de \textbf{calcular con precisión la matriz de homografía} necesaria para transformar un cuadrilátero seleccionado por el usuario en una figura regular, ya sea un cuadrado o un rectángulo con una proporción definida. La detección de puntos de fuga y la aplicación de la transformación proyectiva se ejecutan correctamente incluso en imágenes que presentan perspectivas complejas, garantizando la robustez del proceso.

La interfaz desarrollada facilita una interacción fluida, permitiendo al usuario seleccionar los puntos de interés, ajustar las proporciones deseadas y guardar los resultados obtenidos. Además, la inclusión de instrucciones claras y mecanismos de retroalimentación visual, como la indicación de carga durante el procesamiento, contribuyen a una experiencia de usuario más satisfactoria y comprensible.

El sistema también demuestra una notable adaptabilidad, dado que soporta diversas proporciones, incluyendo valores típicos. Adicionalmente, la integración con el tema del sistema operativo permite que la aplicación se adapte automáticamente al modo claro u oscuro, proporcionando una apariencia coherente con la configuración del usuario.

Finalmente, en términos de rendimiento, la aplicación presenta tiempos de procesamiento aceptables, gracias a la utilización de un método de interpolación sencillo.

\section{Limitaciones Actuales}
\label{sec:limitaciones}
No obstante, existen algunas limitaciones en la implementación actual que deben ser consideradas. En primer lugar, la interpolación basada en el vecino más cercano, aunque simple y rápida, puede generar efectos de aliasing. En imágenes, esto suele manifestarse como bordes o líneas “dentadas” o “escalonadas” en lugar de ser suaves, lo que \textbf{afecta la calidad visual del resultado} final. Futuras versiones priorizarán métodos más avanzados (bilineal/bicúbica) para mejorar la resolución de las imágenes rectificadas.

Además, el \textbf{manejo de errores} aún es limitado. Si el usuario selecciona los puntos en un orden incorrecto o si la geometría de los puntos es inválida, la rectificación puede llevarse a cabo de forma errónea sin ofrecer indicaciones claras que permitan corregir la selección. Esto puede dificultar el uso por parte de usuarios menos experimentados.

Un caso particularmente problemático se presenta cuando se trabaja con imágenes en formato \textbf{horizontal}, es decir, aquellas en las que la anchura del objeto es mayor que su altura. Aunque la aplicación permite introducir razones de aspecto arbitrarias, las instrucciones de uso están orientadas de forma implícita a imágenes verticales (más altas que anchas). Como resultado, el usuario tiende a introducir la razón directamente como alto dividido por ancho, lo cual es correcto, pero al combinarlo con el orden estándar de selección de esquinas (pensado para orientación vertical), el sistema puede interpretar de forma incorrecta la geometría del objeto. El resultado es una imagen de gran dimensión que la aplicación actual no es capaz de procesar al completo, como se ilustra en el ejemplo del cuadro en la Sección~\ref{fg:cuadro-invalido-2}.

La solución está en interpretar la imagen como si estuviera rotada, modificar la razón en consecuencia e invertir el orden de selección de esquinas. Sin embargo, este procedimiento puede resultar poco intuitivo si no se explicita claramente en la documentación o la interfaz, lo que representa una limitación importante desde el punto de vista de la usabilidad.


\section{Mejoras Futuras}

Para superar estas limitaciones y ampliar las capacidades del proyecto, se proponen diversas mejoras para futuras versiones. En términos de \textbf{rendimiento}, se contempla la implementación de métodos de interpolación más avanzados (como la bilineal o la bicúbica mencionadas con anterioridad), que mejorarían la calidad visual de las rectificaciones.

Además, se ha identificado que el método actual de aplicar la homografía de forma \textbf{pixel por pixel} es \textbf{computacionalmente ineficiente}, especialmente para imágenes de alta resolución. Esta ineficiencia afecta el tiempo total de procesamiento y la escalabilidad de la aplicación.

Como propuesta de mejora, se sugiere la \textbf{paralelización de los cálculos}, aprovechando técnicas como la aceleración por GPU mediante tecnologías como \texttt{OpenCL}, así como el uso de bibliotecas optimizadas para procesamiento de imágenes, como \texttt{OpenCV}. 

También se recomienda implementar métodos de interpolación que utilicen operaciones matriciales en lugar de bucles explícitos, lo que puede traducirse en una reducción significativa del tiempo de cómputo y en una mejor calidad visual.

Desde el punto de vista funcional, se planea incorporar la \textbf{detección automática de esquinas} mediante algoritmos, facilitando la selección inicial de puntos por parte del usuario. Además, se contempla ofrecer herramientas para el ajuste manual post-transformación, permitiendo rotaciones y escalados finos para corregir imperfecciones.

Adicionalmente, se prevé implementar un sistema de \textbf{detección automática de orientación} (vertical u horizontal) de la imagen o del objeto seleccionado. Esta funcionalidad permitiría al sistema sugerir —o incluso aplicar internamente— un reordenamiento de los puntos seleccionados y un ajuste coherente de la razón de aspecto introducida. La herramienta podría ofrecer recomendaciones contextuales al usuario, o aplicar automáticamente una rotación previa a la transformación si detecta incoherencias entre la orientación visual de la imagen y los parámetros introducidos, mejorando notablemente la usabilidad y robustez del sistema.


También se espera ampliar el soporte para \textbf{rectificaciones de polígonos irregulares}, como trapezoides, incrementando la versatilidad del sistema.

En lo que respecta a la \textbf{experiencia de usuario}, se propone implementar mejoras que faciliten la interacción y el control sobre el proceso de rectificación. Entre estas mejoras destacan la incorporación de una funcionalidad de deshacer que permita retroceder en los pasos realizados durante la rectificación, la posibilidad de aplicar diversos filtros a las imágenes para mejorar su apariencia o destacar detalles específicos, y la integración de herramientas adicionales para ajustar manualmente aspectos como el brillo, contraste y saturación. Finalmente, se considera ampliar las opciones de exportación, incluyendo formatos profesionales como PDF o SVG.

Estas mejoras tienen como objetivo ofrecer una experiencia más fluida, flexible y personalizada, adaptándose a las necesidades de cada usuario.

\section{Conclusión Final}

\textit{QuadFix} representa la síntesis efectiva entre el rigor matemático de la geometría proyectiva y la potencia de las herramientas modernas de programación y procesamiento de imágenes. A lo largo del proyecto, se ha demostrado que es posible traducir conceptos teóricos complejos en una aplicación intuitiva, accesible y útil para resolver problemas cotidianos, como la distorsión por perspectiva en imágenes de documentos, libros, cuadros o superficies planas en general.

La versión actual cumple los objetivos planteados: permite al usuario seleccionar esquinas, introducir proporciones y obtener una imagen rectificada con precisión. Sin embargo, lo más relevante no es solo lo que \textit{QuadFix} ya hace, sino todo lo que promete. Las mejoras propuestas —como la detección automática de esquinas, la corrección inteligente de orientación o la implementación de interpolaciones más sofisticadas— abren la puerta a convertir esta herramienta en un sistema robusto y profesional, aplicable en sectores tan diversos como la digitalización documental, el diseño gráfico, la arquitectura o la fotogrametría.

Además, la realización de este proyecto nos ha ofrecido una experiencia valiosa a nivel académico y personal. Nos ha permitido enfrentarnos a un ciclo completo de desarrollo, desde la comprensión profunda de los fundamentos teóricos hasta la implementación técnica y el diseño de la interfaz. Este proceso no solo nos ha aportado conocimientos prácticos en geometría proyectiva, álgebra lineal, visión por computador y diseño de interfaces, sino que también nos ha dado una visión clara de lo que implica enfrentarse a un proyecto de mayor envergadura, como puede ser un futuro Trabajo Fin de Grado (TFG). Saber planificar, documentar, tomar decisiones técnicas, iterar sobre el diseño y presentar resultados son competencias que este trabajo nos ha ayudado a desarrollar y que sin duda serán fundamentales en nuestra formación profesional.

En definitiva, \textit{QuadFix} no es solo una solución a un problema concreto. Es una demostración de que, cuando las matemáticas, la tecnología y el diseño de experiencia se alinean, pueden transformar tareas cotidianas en procesos eficientes, automatizados y elegantes. Y este es solo el comienzo.
